\newpage
\section{D�roulement du projet}
\label{sec:deroulement}

\noindent Dans cette section, nous d�crivons comment la r�alisation du projet s'est d�roul�e au sein de l'�quipe de projet. 

\subsection{Synchronisation du travail}
\label{sec:synchro}

\paragraph{}Afin de pouvoir travailler sur ce projet nous avons utilis� la plateforme Github

\subsection{R�partition du travail}
\label{sec:repartition}

\begin{table}[H]
\centering
\begin{tabular} {|p{7cm}|p{7cm}|}
\hline{\centering}
\bf Valentin & \bf Quentin \\
\hline
ALU & Banc de Registre \\
\hline
Extension & Unit� de Controle \\
\hline
Compte rendu & Unit� d'Adressage\\
\hline
\end{tabular}
\caption{R�partition des t�ches}
\label{tab:repartition}
\end{table}

\subsection{Probl�mes rencontr�s}
\label{sec:problemes}
\begin{itemize}
\item L'Impl�mentation de l'op�ration soustraction
\item R�alisation de l'unit� d'adressage
\item La compr�hension de l'utilit� d'un registre d'adresse l'adresse de donn�es
\item La compr�hension de certains signaux de sorties de l'UC (Fetch)
\item La bascule D qui cr�ait un probl�me d'oscillation (Nous avons donc utilis� les registres fourni par Logisim)
\end{itemize}