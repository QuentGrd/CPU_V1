\section{Sp�cification du processeur}
\label{sec:specification}

\paragraph{}Dans cette section, nous pr�sentons la sp�cification du processeur r�alis�. Pour cela, nous allons d�crire chaque �l�ment qui le compose. 
\subsection{L'ALU}
\label{sec:spec1}

\paragraph{} Une ALU est une Unit� Arithm�tique et Logique. Elle permet de r�aliser des op�rations sur des op�randes pr�sentes � ses entr�s. Pour ce projet, le but est de r�aliser une ALU capable d'effectuer 8 op�arations diff�rentes.

\subsection{Le banc de registre}
\label{sec:spec2}

\paragraph{}Un banc de registres est une m�moire interne au processeur, dans laquelle sont rassembl�s certains des registres du processeur. 
Ce banc de registre contient : 
\begin{itemize}
\item 1 entr�e sur 4 bits
\item 2 sorties sur 4 bits
\item 3 signaux de controle (lecture sur X, lecture sur Y et ecriture sur un registre)
\item une horloge
\end{itemize}

\subsection{L'unit� d'adressage}
\label{sec:spec3}

\paragraph{} 

\subsection{L'unit� de contr�le}
\label{sec:spec4}

\paragraph{} Le principe de fonctionnement d'une unit� de contr�le est de lire le contenu du registre d'instruction, de le d�coder et d'en d�duire le positionnement des diff�rents signaux de l'architecture (ALU, sources, destinations, acc�s m�moire).